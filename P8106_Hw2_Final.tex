\documentclass[]{article}
\usepackage{lmodern}
\usepackage{amssymb,amsmath}
\usepackage{ifxetex,ifluatex}
\usepackage{fixltx2e} % provides \textsubscript
\ifnum 0\ifxetex 1\fi\ifluatex 1\fi=0 % if pdftex
  \usepackage[T1]{fontenc}
  \usepackage[utf8]{inputenc}
\else % if luatex or xelatex
  \ifxetex
    \usepackage{mathspec}
  \else
    \usepackage{fontspec}
  \fi
  \defaultfontfeatures{Ligatures=TeX,Scale=MatchLowercase}
\fi
% use upquote if available, for straight quotes in verbatim environments
\IfFileExists{upquote.sty}{\usepackage{upquote}}{}
% use microtype if available
\IfFileExists{microtype.sty}{%
\usepackage{microtype}
\UseMicrotypeSet[protrusion]{basicmath} % disable protrusion for tt fonts
}{}
\usepackage[margin=1in]{geometry}
\usepackage{hyperref}
\hypersetup{unicode=true,
            pdftitle={P8106\_HW2},
            pdfauthor={Ekta Chaudhary},
            pdfborder={0 0 0},
            breaklinks=true}
\urlstyle{same}  % don't use monospace font for urls
\usepackage{color}
\usepackage{fancyvrb}
\newcommand{\VerbBar}{|}
\newcommand{\VERB}{\Verb[commandchars=\\\{\}]}
\DefineVerbatimEnvironment{Highlighting}{Verbatim}{commandchars=\\\{\}}
% Add ',fontsize=\small' for more characters per line
\usepackage{framed}
\definecolor{shadecolor}{RGB}{248,248,248}
\newenvironment{Shaded}{\begin{snugshade}}{\end{snugshade}}
\newcommand{\AlertTok}[1]{\textcolor[rgb]{0.94,0.16,0.16}{#1}}
\newcommand{\AnnotationTok}[1]{\textcolor[rgb]{0.56,0.35,0.01}{\textbf{\textit{#1}}}}
\newcommand{\AttributeTok}[1]{\textcolor[rgb]{0.77,0.63,0.00}{#1}}
\newcommand{\BaseNTok}[1]{\textcolor[rgb]{0.00,0.00,0.81}{#1}}
\newcommand{\BuiltInTok}[1]{#1}
\newcommand{\CharTok}[1]{\textcolor[rgb]{0.31,0.60,0.02}{#1}}
\newcommand{\CommentTok}[1]{\textcolor[rgb]{0.56,0.35,0.01}{\textit{#1}}}
\newcommand{\CommentVarTok}[1]{\textcolor[rgb]{0.56,0.35,0.01}{\textbf{\textit{#1}}}}
\newcommand{\ConstantTok}[1]{\textcolor[rgb]{0.00,0.00,0.00}{#1}}
\newcommand{\ControlFlowTok}[1]{\textcolor[rgb]{0.13,0.29,0.53}{\textbf{#1}}}
\newcommand{\DataTypeTok}[1]{\textcolor[rgb]{0.13,0.29,0.53}{#1}}
\newcommand{\DecValTok}[1]{\textcolor[rgb]{0.00,0.00,0.81}{#1}}
\newcommand{\DocumentationTok}[1]{\textcolor[rgb]{0.56,0.35,0.01}{\textbf{\textit{#1}}}}
\newcommand{\ErrorTok}[1]{\textcolor[rgb]{0.64,0.00,0.00}{\textbf{#1}}}
\newcommand{\ExtensionTok}[1]{#1}
\newcommand{\FloatTok}[1]{\textcolor[rgb]{0.00,0.00,0.81}{#1}}
\newcommand{\FunctionTok}[1]{\textcolor[rgb]{0.00,0.00,0.00}{#1}}
\newcommand{\ImportTok}[1]{#1}
\newcommand{\InformationTok}[1]{\textcolor[rgb]{0.56,0.35,0.01}{\textbf{\textit{#1}}}}
\newcommand{\KeywordTok}[1]{\textcolor[rgb]{0.13,0.29,0.53}{\textbf{#1}}}
\newcommand{\NormalTok}[1]{#1}
\newcommand{\OperatorTok}[1]{\textcolor[rgb]{0.81,0.36,0.00}{\textbf{#1}}}
\newcommand{\OtherTok}[1]{\textcolor[rgb]{0.56,0.35,0.01}{#1}}
\newcommand{\PreprocessorTok}[1]{\textcolor[rgb]{0.56,0.35,0.01}{\textit{#1}}}
\newcommand{\RegionMarkerTok}[1]{#1}
\newcommand{\SpecialCharTok}[1]{\textcolor[rgb]{0.00,0.00,0.00}{#1}}
\newcommand{\SpecialStringTok}[1]{\textcolor[rgb]{0.31,0.60,0.02}{#1}}
\newcommand{\StringTok}[1]{\textcolor[rgb]{0.31,0.60,0.02}{#1}}
\newcommand{\VariableTok}[1]{\textcolor[rgb]{0.00,0.00,0.00}{#1}}
\newcommand{\VerbatimStringTok}[1]{\textcolor[rgb]{0.31,0.60,0.02}{#1}}
\newcommand{\WarningTok}[1]{\textcolor[rgb]{0.56,0.35,0.01}{\textbf{\textit{#1}}}}
\usepackage{graphicx,grffile}
\makeatletter
\def\maxwidth{\ifdim\Gin@nat@width>\linewidth\linewidth\else\Gin@nat@width\fi}
\def\maxheight{\ifdim\Gin@nat@height>\textheight\textheight\else\Gin@nat@height\fi}
\makeatother
% Scale images if necessary, so that they will not overflow the page
% margins by default, and it is still possible to overwrite the defaults
% using explicit options in \includegraphics[width, height, ...]{}
\setkeys{Gin}{width=\maxwidth,height=\maxheight,keepaspectratio}
\IfFileExists{parskip.sty}{%
\usepackage{parskip}
}{% else
\setlength{\parindent}{0pt}
\setlength{\parskip}{6pt plus 2pt minus 1pt}
}
\setlength{\emergencystretch}{3em}  % prevent overfull lines
\providecommand{\tightlist}{%
  \setlength{\itemsep}{0pt}\setlength{\parskip}{0pt}}
\setcounter{secnumdepth}{0}
% Redefines (sub)paragraphs to behave more like sections
\ifx\paragraph\undefined\else
\let\oldparagraph\paragraph
\renewcommand{\paragraph}[1]{\oldparagraph{#1}\mbox{}}
\fi
\ifx\subparagraph\undefined\else
\let\oldsubparagraph\subparagraph
\renewcommand{\subparagraph}[1]{\oldsubparagraph{#1}\mbox{}}
\fi

%%% Use protect on footnotes to avoid problems with footnotes in titles
\let\rmarkdownfootnote\footnote%
\def\footnote{\protect\rmarkdownfootnote}

%%% Change title format to be more compact
\usepackage{titling}

% Create subtitle command for use in maketitle
\providecommand{\subtitle}[1]{
  \posttitle{
    \begin{center}\large#1\end{center}
    }
}

\setlength{\droptitle}{-2em}

  \title{P8106\_HW2}
    \pretitle{\vspace{\droptitle}\centering\huge}
  \posttitle{\par}
    \author{Ekta Chaudhary}
    \preauthor{\centering\large\emph}
  \postauthor{\par}
      \predate{\centering\large\emph}
  \postdate{\par}
    \date{3/22/2020}


\begin{document}
\maketitle

\begin{Shaded}
\begin{Highlighting}[]
\KeywordTok{library}\NormalTok{(tidyverse)}
\KeywordTok{library}\NormalTok{(caret)}
\KeywordTok{library}\NormalTok{(ModelMetrics)}
\KeywordTok{library}\NormalTok{(glmnet)}
\KeywordTok{library}\NormalTok{(gam)}
\KeywordTok{library}\NormalTok{(boot)}
\KeywordTok{library}\NormalTok{(mgcv)}
\KeywordTok{library}\NormalTok{(splines)}
\KeywordTok{library}\NormalTok{(ggplot2)}
\KeywordTok{library}\NormalTok{(lasso2)}
\KeywordTok{library}\NormalTok{(pdp)}
\KeywordTok{library}\NormalTok{(earth)}
\end{Highlighting}
\end{Shaded}

Reading the Datasets

\begin{Shaded}
\begin{Highlighting}[]
\NormalTok{data =}\StringTok{ }
\StringTok{  }\KeywordTok{read_csv}\NormalTok{(}\StringTok{'./data/College.csv'}\NormalTok{) }\OperatorTok
\KeywordTok{select}\NormalTok{(}\OperatorTok{-}\NormalTok{College)}
\NormalTok{data_}\DecValTok{1}\NormalTok{ =}
\StringTok{  }\NormalTok{data[}\OperatorTok{-}\DecValTok{125}\NormalTok{,]}
\NormalTok{data_}\DecValTok{2}\NormalTok{ =}
\StringTok{  }\NormalTok{data[}\DecValTok{125}\NormalTok{,]}
\end{Highlighting}
\end{Shaded}

\begin{Shaded}
\begin{Highlighting}[]
\NormalTok{x <-}\StringTok{ }\KeywordTok{model.matrix}\NormalTok{(Outstate}\OperatorTok{~}\NormalTok{.,data_}\DecValTok{1}\NormalTok{)[,}\OperatorTok{-}\DecValTok{1}\NormalTok{]}
\NormalTok{y <-}\StringTok{ }\NormalTok{data_}\DecValTok{1}\OperatorTok{$}\NormalTok{Outstate}
\end{Highlighting}
\end{Shaded}

\hypertarget{a-create-scatter-plots-of-response-vs.predictors.}{%
\section{(a) Create scatter plots of response
vs.~predictors.}\label{a-create-scatter-plots-of-response-vs.predictors.}}

\begin{Shaded}
\begin{Highlighting}[]
\NormalTok{theme1 <-}\StringTok{ }\KeywordTok{trellis.par.get}\NormalTok{()}
\NormalTok{theme1}\OperatorTok{$}\NormalTok{plot.symbol}\OperatorTok{$}\NormalTok{col <-}\StringTok{ }\KeywordTok{rgb}\NormalTok{(.}\DecValTok{3}\NormalTok{, }\FloatTok{.5}\NormalTok{, }\FloatTok{.2}\NormalTok{, }\FloatTok{.5}\NormalTok{)}
\NormalTok{theme1}\OperatorTok{$}\NormalTok{plot.symbol}\OperatorTok{$}\NormalTok{pch <-}\StringTok{ }\DecValTok{18}
\NormalTok{theme1}\OperatorTok{$}\NormalTok{plot.line}\OperatorTok{$}\NormalTok{col <-}\StringTok{ }\KeywordTok{rgb}\NormalTok{(.}\DecValTok{8}\NormalTok{, }\FloatTok{.1}\NormalTok{, }\FloatTok{.1}\NormalTok{, }\DecValTok{1}\NormalTok{)}
\NormalTok{theme1}\OperatorTok{$}\NormalTok{plot.line}\OperatorTok{$}\NormalTok{lwd <-}\StringTok{ }\DecValTok{2}
\NormalTok{theme1}\OperatorTok{$}\NormalTok{strip.background}\OperatorTok{$}\NormalTok{col <-}\StringTok{ }\KeywordTok{rgb}\NormalTok{(.}\DecValTok{0}\NormalTok{, }\FloatTok{.2}\NormalTok{, }\FloatTok{.6}\NormalTok{, }\FloatTok{.2}\NormalTok{)}
\KeywordTok{trellis.par.set}\NormalTok{(theme1)}
\KeywordTok{featurePlot}\NormalTok{(x, y, }\DataTypeTok{plot =} \StringTok{"scatter"}\NormalTok{, }\DataTypeTok{labels =} \KeywordTok{c}\NormalTok{(}\StringTok{""}\NormalTok{,}\StringTok{"Y"}\NormalTok{),}
            \DataTypeTok{type =} \KeywordTok{c}\NormalTok{(}\StringTok{"p"}\NormalTok{), }\DataTypeTok{layout =} \KeywordTok{c}\NormalTok{(}\DecValTok{4}\NormalTok{, }\DecValTok{2}\NormalTok{))}
\end{Highlighting}
\end{Shaded}

\includegraphics{P8106_Hw2_Final_files/figure-latex/unnamed-chunk-4-1.pdf}
\includegraphics{P8106_Hw2_Final_files/figure-latex/unnamed-chunk-4-2.pdf}

\hypertarget{b-fit-a-smoothing-spline-model-using-terminal-as-the-only-predictor-of-outstate-for-a-range-of-degrees-of-freedom-as-well-as-the-degree-of-freedom-obtained-by-generalized-cross--validation-and-plot-the-resulting-fits.-describe-the-results-obtained.}{%
\section{b) Fit a smoothing spline model using Terminal as the only
predictor of Outstate for a range of degrees of freedom, as well as the
degree of freedom obtained by generalized cross- validation, and plot
the resulting fits. Describe the results
obtained.}\label{b-fit-a-smoothing-spline-model-using-terminal-as-the-only-predictor-of-outstate-for-a-range-of-degrees-of-freedom-as-well-as-the-degree-of-freedom-obtained-by-generalized-cross--validation-and-plot-the-resulting-fits.-describe-the-results-obtained.}}

\begin{Shaded}
\begin{Highlighting}[]
\NormalTok{Terminallims <-}\StringTok{ }\KeywordTok{range}\NormalTok{(data_}\DecValTok{1}\OperatorTok{$}\NormalTok{Terminal)}
\NormalTok{Terminal.grid <-}\StringTok{ }\KeywordTok{seq}\NormalTok{(}\DataTypeTok{from =}\NormalTok{ Terminallims[}\DecValTok{1}\NormalTok{],}\DataTypeTok{to =}\NormalTok{ Terminallims[}\DecValTok{2}\NormalTok{])}

\NormalTok{fit.ss <-}\StringTok{ }\KeywordTok{smooth.spline}\NormalTok{(data_}\DecValTok{1}\OperatorTok{$}\NormalTok{Terminal, data_}\DecValTok{1}\OperatorTok{$}\NormalTok{Outstate)}
\NormalTok{fit.ss}\OperatorTok{$}\NormalTok{df}
\end{Highlighting}
\end{Shaded}

\begin{verbatim}
## [1] 4.468629
\end{verbatim}

\begin{Shaded}
\begin{Highlighting}[]
\NormalTok{pred.ss <-}\StringTok{ }\KeywordTok{predict}\NormalTok{(fit.ss,}
                   \DataTypeTok{x =}\NormalTok{ Terminal.grid)}

\NormalTok{pred.ss.df <-}\StringTok{ }\KeywordTok{data.frame}\NormalTok{(}\DataTypeTok{pred =}\NormalTok{ pred.ss}\OperatorTok{$}\NormalTok{y,}
                         \DataTypeTok{Terminal =}\NormalTok{ Terminal.grid)}

\NormalTok{p <-}\StringTok{ }\KeywordTok{ggplot}\NormalTok{(}\DataTypeTok{data =}\NormalTok{ data_}\DecValTok{1}\NormalTok{, }\KeywordTok{aes}\NormalTok{(}\DataTypeTok{x =}\NormalTok{ Terminal, }\DataTypeTok{y =}\NormalTok{ Outstate)) }\OperatorTok{+}
\StringTok{  }\KeywordTok{geom_point}\NormalTok{(}\DataTypeTok{color =} \KeywordTok{rgb}\NormalTok{(.}\DecValTok{2}\NormalTok{, }\FloatTok{.4}\NormalTok{, }\FloatTok{.2}\NormalTok{, }\FloatTok{.5}\NormalTok{))}
\NormalTok{p }\OperatorTok{+}\StringTok{ }\KeywordTok{geom_line}\NormalTok{(}\KeywordTok{aes}\NormalTok{(}\DataTypeTok{x =}\NormalTok{ Terminal, }\DataTypeTok{y =}\NormalTok{ pred), }\DataTypeTok{data =}\NormalTok{ pred.ss.df, }
              \DataTypeTok{color =} \KeywordTok{rgb}\NormalTok{(.}\DecValTok{8}\NormalTok{, }\FloatTok{.1}\NormalTok{, }\FloatTok{.1}\NormalTok{, }\DecValTok{1}\NormalTok{)) }\OperatorTok{+}\StringTok{ }\KeywordTok{theme_bw}\NormalTok{() }\OperatorTok{+}\StringTok{ }
\StringTok{  }\KeywordTok{theme}\NormalTok{(}\DataTypeTok{plot.title =} \KeywordTok{element_text}\NormalTok{(}\DataTypeTok{hjust =} \FloatTok{0.5}\NormalTok{))}
\end{Highlighting}
\end{Shaded}

\includegraphics{P8106_Hw2_Final_files/figure-latex/unnamed-chunk-5-1.pdf}

The function \texttt{smooth.spline()} can be used to fit smoothing
spline models. Generalized cross-validation is used to select the degree
of freedom. The degree of freedom obtained by generalized cross-
validation is \textbf{4.468629}.

\begin{Shaded}
\begin{Highlighting}[]
\KeywordTok{par}\NormalTok{(}\DataTypeTok{mfrow =} \KeywordTok{c}\NormalTok{(}\DecValTok{3}\NormalTok{,}\DecValTok{3}\NormalTok{)) }\CommentTok{# 3 x 3 grid}
\NormalTok{all.dfs =}\StringTok{ }\KeywordTok{rep}\NormalTok{(}\OtherTok{NA}\NormalTok{, }\DecValTok{9}\NormalTok{)}
\ControlFlowTok{for}\NormalTok{ (i }\ControlFlowTok{in} \DecValTok{2}\OperatorTok{:}\DecValTok{20}\NormalTok{) \{}
\NormalTok{  fit.ss =}\StringTok{ }\KeywordTok{smooth.spline}\NormalTok{(data_}\DecValTok{1}\OperatorTok{$}\NormalTok{Terminal, data_}\DecValTok{1}\OperatorTok{$}\NormalTok{Outstate, }\DataTypeTok{df =}\NormalTok{ i)}
  
\NormalTok{  pred.ss <-}\StringTok{ }\KeywordTok{predict}\NormalTok{(fit.ss, }\DataTypeTok{x =}\NormalTok{ Terminal.grid)}
  
  \KeywordTok{plot}\NormalTok{(data_}\DecValTok{1}\OperatorTok{$}\NormalTok{Terminal, data_}\DecValTok{1}\OperatorTok{$}\NormalTok{Outstate, }\DataTypeTok{cex =} \FloatTok{.5}\NormalTok{, }\DataTypeTok{col =} \StringTok{"darkgrey"}\NormalTok{)}
  \KeywordTok{title}\NormalTok{(}\KeywordTok{paste}\NormalTok{(}\StringTok{"Degrees of freedom = "}\NormalTok{, }\KeywordTok{round}\NormalTok{(fit.ss}\OperatorTok{$}\NormalTok{df)),  }\DataTypeTok{outer =}\NormalTok{ F)}
  \KeywordTok{lines}\NormalTok{(Terminal.grid, pred.ss}\OperatorTok{$}\NormalTok{y, }\DataTypeTok{lwd =} \DecValTok{2}\NormalTok{, }\DataTypeTok{col =} \StringTok{"blue"}\NormalTok{)}
\NormalTok{\}}
\end{Highlighting}
\end{Shaded}

\includegraphics{P8106_Hw2_Final_files/figure-latex/unnamed-chunk-6-1.pdf}
\includegraphics{P8106_Hw2_Final_files/figure-latex/unnamed-chunk-6-2.pdf}
\includegraphics{P8106_Hw2_Final_files/figure-latex/unnamed-chunk-6-3.pdf}
I have picked a range of degrees of freedom from 2 to 20. As it can be
seen from the plots, when the degree of freedom is 2, the model is
linear and when the df increases the model gets wiggly.

\hypertarget{c-fit-a-generalized-additive-model-gam-using-all-the-predictors.-plot-the-results-and-explain-your-findings.}{%
\section{c) Fit a generalized additive model (GAM) using all the
predictors. Plot the results and explain your
findings.}\label{c-fit-a-generalized-additive-model-gam-using-all-the-predictors.-plot-the-results-and-explain-your-findings.}}

\begin{Shaded}
\begin{Highlighting}[]
\NormalTok{gam.m1 =}\StringTok{ }\KeywordTok{gam}\NormalTok{(}
\NormalTok{  Outstate}\OperatorTok{~}\StringTok{ }\NormalTok{Apps }\OperatorTok{+}\StringTok{ }\NormalTok{Accept }\OperatorTok{+}\StringTok{ }\NormalTok{Enroll }\OperatorTok{+}\StringTok{ }\NormalTok{Top10perc }\OperatorTok{+}\StringTok{ }\NormalTok{Top25perc }\OperatorTok{+}\StringTok{ }\NormalTok{F.Undergrad }\OperatorTok{+}\StringTok{ }\NormalTok{P.Undergrad }\OperatorTok{+}\StringTok{ }\NormalTok{Room.Board }\OperatorTok{+}\StringTok{ }\NormalTok{Books }\OperatorTok{+}\StringTok{ }\NormalTok{Personal }\OperatorTok{+}\StringTok{ }\NormalTok{PhD }\OperatorTok{+}\StringTok{ }\NormalTok{Terminal }\OperatorTok{+}\StringTok{ }\NormalTok{S.F.Ratio }\OperatorTok{+}\StringTok{ }\NormalTok{perc.alumni }\OperatorTok{+}\StringTok{ }\NormalTok{Expend }\OperatorTok{+}\StringTok{ }\NormalTok{Grad.Rate, }
  \DataTypeTok{data =}\NormalTok{ data_}\DecValTok{1}\NormalTok{)}
\NormalTok{gam.m2 =}\StringTok{ }\KeywordTok{gam}\NormalTok{(}
\NormalTok{  Outstate}\OperatorTok{~}\StringTok{ }\NormalTok{Apps }\OperatorTok{+}\StringTok{ }\NormalTok{Accept }\OperatorTok{+}\StringTok{ }\NormalTok{Enroll }\OperatorTok{+}\StringTok{ }\NormalTok{Top10perc }\OperatorTok{+}\StringTok{ }\NormalTok{Top25perc }\OperatorTok{+}\StringTok{ }\NormalTok{F.Undergrad }\OperatorTok{+}\StringTok{ }\NormalTok{P.Undergrad }\OperatorTok{+}\StringTok{ }\NormalTok{Room.Board }\OperatorTok{+}\StringTok{ }\NormalTok{Books }\OperatorTok{+}\StringTok{ }\NormalTok{Personal }\OperatorTok{+}\StringTok{ }\NormalTok{PhD }\OperatorTok{+}\StringTok{ }\KeywordTok{s}\NormalTok{(Terminal) }\OperatorTok{+}\StringTok{ }\NormalTok{S.F.Ratio }\OperatorTok{+}\StringTok{ }\NormalTok{perc.alumni }\OperatorTok{+}\StringTok{ }\NormalTok{Expend }\OperatorTok{+}\StringTok{ }\NormalTok{Grad.Rate,}
  \DataTypeTok{data =}\NormalTok{ data_}\DecValTok{1}\NormalTok{)}
\NormalTok{gam.m3 =}\StringTok{ }\KeywordTok{gam}\NormalTok{(}
\NormalTok{  Outstate}\OperatorTok{~}\StringTok{ }\NormalTok{Apps }\OperatorTok{+}\StringTok{ }\NormalTok{Accept }\OperatorTok{+}\StringTok{ }\NormalTok{Enroll }\OperatorTok{+}\StringTok{ }\NormalTok{Top10perc }\OperatorTok{+}\StringTok{ }\NormalTok{Top25perc }\OperatorTok{+}\StringTok{ }\NormalTok{F.Undergrad }\OperatorTok{+}\StringTok{ }\NormalTok{P.Undergrad }\OperatorTok{+}\StringTok{ }\KeywordTok{te}\NormalTok{(Room.Board) }\OperatorTok{+}\StringTok{ }\KeywordTok{te}\NormalTok{(Personal) }\OperatorTok{+}\StringTok{ }\NormalTok{Books }\OperatorTok{+}\StringTok{ }\NormalTok{PhD }\OperatorTok{+}\StringTok{ }\KeywordTok{s}\NormalTok{(Terminal) }\OperatorTok{+}\StringTok{ }\NormalTok{S.F.Ratio }\OperatorTok{+}\StringTok{ }\NormalTok{perc.alumni }\OperatorTok{+}\StringTok{ }\NormalTok{Expend }\OperatorTok{+}\StringTok{ }\NormalTok{Grad.Rate, }\DataTypeTok{data =}\NormalTok{ data_}\DecValTok{1}\NormalTok{)}

\KeywordTok{anova}\NormalTok{(gam.m1, gam.m2, gam.m3, }\DataTypeTok{test =} \StringTok{"F"}\NormalTok{)}
\end{Highlighting}
\end{Shaded}

\begin{verbatim}
## Analysis of Deviance Table
## 
## Model 1: Outstate ~ Apps + Accept + Enroll + Top10perc + Top25perc + F.Undergrad + 
##     P.Undergrad + Room.Board + Books + Personal + PhD + Terminal + 
##     S.F.Ratio + perc.alumni + Expend + Grad.Rate
## Model 2: Outstate ~ Apps + Accept + Enroll + Top10perc + Top25perc + F.Undergrad + 
##     P.Undergrad + Room.Board + Books + Personal + PhD + s(Terminal) + 
##     S.F.Ratio + perc.alumni + Expend + Grad.Rate
## Model 3: Outstate ~ Apps + Accept + Enroll + Top10perc + Top25perc + F.Undergrad + 
##     P.Undergrad + te(Room.Board) + te(Personal) + Books + PhD + 
##     s(Terminal) + S.F.Ratio + perc.alumni + Expend + Grad.Rate
##   Resid. Df Resid. Dev     Df Deviance      F    Pr(>F)    
## 1    547.00 2092185295                                     
## 2    542.37 2026858216 4.6295 65327078 3.9364  0.002202 ** 
## 3    537.48 1933201900 4.8882 93656316 5.3448 9.659e-05 ***
## ---
## Signif. codes:  0 '***' 0.001 '**' 0.01 '*' 0.05 '.' 0.1 ' ' 1
\end{verbatim}

Looking at the p-values from the ANOVA test, Model 3 appears to be the
best fitting model.

\begin{Shaded}
\begin{Highlighting}[]
\KeywordTok{plot}\NormalTok{(gam.m2)}
\end{Highlighting}
\end{Shaded}

\includegraphics{P8106_Hw2_Final_files/figure-latex/unnamed-chunk-8-1.pdf}

\begin{Shaded}
\begin{Highlighting}[]
\KeywordTok{plot}\NormalTok{(gam.m3)}
\end{Highlighting}
\end{Shaded}

\includegraphics{P8106_Hw2_Final_files/figure-latex/unnamed-chunk-8-2.pdf}
\includegraphics{P8106_Hw2_Final_files/figure-latex/unnamed-chunk-8-3.pdf}
\includegraphics{P8106_Hw2_Final_files/figure-latex/unnamed-chunk-8-4.pdf}

\begin{Shaded}
\begin{Highlighting}[]
\KeywordTok{vis.gam}\NormalTok{(gam.m3, }\DataTypeTok{view =} \KeywordTok{c}\NormalTok{(}\StringTok{"Room.Board"}\NormalTok{,}\StringTok{"Personal"}\NormalTok{),}\DataTypeTok{plot.type =} \StringTok{"contour"}\NormalTok{, }\DataTypeTok{color =} \StringTok{"topo"}\NormalTok{)}
\end{Highlighting}
\end{Shaded}

\includegraphics{P8106_Hw2_Final_files/figure-latex/unnamed-chunk-8-5.pdf}

\begin{enumerate}
\def\labelenumi{(\alph{enumi})}
\setcounter{enumi}{3}
\tightlist
\item
  Fit a multivariate adaptive regression spline (MARS) model using all
  the predictors. Report the final model. Present the partial dependence
  plot of an arbitrary predictor in your final model.
\end{enumerate}

\begin{Shaded}
\begin{Highlighting}[]
\NormalTok{ctrl1 <-}\StringTok{ }\KeywordTok{trainControl}\NormalTok{(}\DataTypeTok{method =} \StringTok{"cv"}\NormalTok{, }\DataTypeTok{number =} \DecValTok{10}\NormalTok{)}
\NormalTok{mars_grid <-}\StringTok{ }\KeywordTok{expand.grid}\NormalTok{(}\DataTypeTok{degree =} \DecValTok{1}\OperatorTok{:}\DecValTok{2}\NormalTok{, }
                         \DataTypeTok{nprune =} \DecValTok{2}\OperatorTok{:}\DecValTok{10}\NormalTok{)}

\KeywordTok{set.seed}\NormalTok{(}\DecValTok{2}\NormalTok{)}
\NormalTok{mars.fit <-}\StringTok{ }\KeywordTok{train}\NormalTok{(x, y,}
                  \DataTypeTok{method =} \StringTok{"earth"}\NormalTok{,}
                  \DataTypeTok{tuneGrid =}\NormalTok{ mars_grid,}
                  \DataTypeTok{trControl =}\NormalTok{ ctrl1)}

\KeywordTok{ggplot}\NormalTok{(mars.fit)}
\end{Highlighting}
\end{Shaded}

\includegraphics{P8106_Hw2_Final_files/figure-latex/unnamed-chunk-9-1.pdf}

\begin{Shaded}
\begin{Highlighting}[]
\NormalTok{mars.fit}\OperatorTok{$}\NormalTok{bestTune}
\end{Highlighting}
\end{Shaded}

\begin{verbatim}
##    nprune degree
## 18     10      2
\end{verbatim}

\begin{Shaded}
\begin{Highlighting}[]
\KeywordTok{coef}\NormalTok{(mars.fit}\OperatorTok{$}\NormalTok{finalModel) }
\end{Highlighting}
\end{Shaded}

\begin{verbatim}
##                        (Intercept)                    h(15365-Expend) 
##                       1.602840e+04                      -6.313124e-01 
##                 h(4450-Room.Board)                  h(perc.alumni-22) 
##                      -1.651132e+00                       9.956084e+01 
##                  h(22-perc.alumni) h(1546-Accept) * h(perc.alumni-22) 
##                      -1.017750e+02                      -1.503182e-01 
##                          h(PhD-81)    h(F.Undergrad-1355) * h(PhD-45) 
##                       1.173551e+02                      -1.052312e-02 
##    h(F.Undergrad-1355) * h(45-PhD)                     h(Accept-2342) 
##                      -8.322623e-02                       7.016650e-01
\end{verbatim}

Present the partial dependence plot of an arbitrary predictor in your
final model.

\begin{Shaded}
\begin{Highlighting}[]
\NormalTok{p1 <-}\StringTok{ }\KeywordTok{partial}\NormalTok{(mars.fit, }\DataTypeTok{pred.var =} \KeywordTok{c}\NormalTok{(}\StringTok{"Room.Board"}\NormalTok{), }\DataTypeTok{grid.resolution =} \DecValTok{10}\NormalTok{) }\OperatorTok\StringTok{ }\KeywordTok{autoplot}\NormalTok{()}

\NormalTok{p2 <-}\StringTok{ }\KeywordTok{partial}\NormalTok{(mars.fit, }\DataTypeTok{pred.var =} \KeywordTok{c}\NormalTok{(}\StringTok{"Room.Board"}\NormalTok{, }\StringTok{"perc.alumni"}\NormalTok{), }\DataTypeTok{grid.resolution =} \DecValTok{10}\NormalTok{) }\OperatorTok
\StringTok{      }\KeywordTok{plotPartial}\NormalTok{(}\DataTypeTok{levelplot =} \OtherTok{FALSE}\NormalTok{, }\DataTypeTok{zlab =} \StringTok{"yhat"}\NormalTok{, }\DataTypeTok{drape =} \OtherTok{TRUE}\NormalTok{, }
                  \DataTypeTok{screen =} \KeywordTok{list}\NormalTok{(}\DataTypeTok{z =} \DecValTok{20}\NormalTok{, }\DataTypeTok{x =} \DecValTok{-60}\NormalTok{))}

\KeywordTok{grid.arrange}\NormalTok{(p1, p2, }\DataTypeTok{ncol =} \DecValTok{2}\NormalTok{)}
\end{Highlighting}
\end{Shaded}

\includegraphics{P8106_Hw2_Final_files/figure-latex/unnamed-chunk-10-1.pdf}

\begin{Shaded}
\begin{Highlighting}[]
\NormalTok{pred.gam <-}\StringTok{ }\KeywordTok{predict}\NormalTok{(gam.m2, }\DataTypeTok{newdata =}\NormalTok{ data_}\DecValTok{2}\NormalTok{)}
\NormalTok{pred.mars <-}\StringTok{ }\KeywordTok{predict}\NormalTok{(mars.fit, }\DataTypeTok{newdata =}\NormalTok{ data_}\DecValTok{2}\NormalTok{)}
\NormalTok{pred.gam}
\end{Highlighting}
\end{Shaded}

\begin{verbatim}
##        1 
## 19406.71
\end{verbatim}

\begin{Shaded}
\begin{Highlighting}[]
\NormalTok{pred.mars}
\end{Highlighting}
\end{Shaded}

\begin{verbatim}
##             y
## [1,] 16698.41
\end{verbatim}


\end{document}
